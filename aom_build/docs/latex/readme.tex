
\begin{DoxyCodeInclude}
1 # AV1 Codec Library
2 
3 ## Contents
4 1. [Building the lib and applications](#building-the-library-and-applications)
5     - [Prerequisites](#prerequisites)
6     - [Get the code](#get-the-code)
7     - [Basics](#basic-build)
8     - [Configuration options](#configuration-options)
9     - [Dylib builds](#dylib-builds)
10     - [Debugging](#debugging)
11     - [Cross compiling](#cross-compiling)
12     - [Sanitizer support](#sanitizers)
13     - [MSVC builds](#microsoft-visual-studio-builds)
14     - [Xcode builds](#xcode-builds)
15     - [Emscripten builds](#emscripten-builds)
16     - [Extra Build Flags](#extra-build-flags)
17 2. [Testing the library](#testing-the-av1-codec)
18     - [Basics](#testing-basics)
19         - [Unit tests](#1\_unit-tests)
20         - [Example tests](#2\_example-tests)
21         - [Encoder tests](#3\_encoder-tests)
22     - [IDE hosted tests](#ide-hosted-tests)
23     - [Downloading test data](#downloading-the-test-data)
24     - [Adding a new test data file](#adding-a-new-test-data-file)
25     - [Additional test data](#additional-test-data)
26     - [Sharded testing](#sharded-testing)
27         - [Running tests directly](#1\_running-test\_libaom-directly)
28         - [Running tests via CMake](#2\_running-the-tests-via-the-cmake-build)
29 3. [Coding style](#coding-style)
30 4. [Submitting patches](#submitting-patches)
31     - [Login cookie](#login-cookie)
32     - [Contributor agreement](#contributor-agreement)
33     - [Testing your code](#testing-your-code)
34     - [Commit message hook](#commit-message-hook)
35     - [Upload your change](#upload-your-change)
36     - [Incorporating Reviewer Comments](#incorporating-reviewer-comments)
37     - [Submitting your change](#submitting-your-change)
38     - [Viewing change status](#viewing-the-status-of-uploaded-changes)
39 5. [Support](#support)
40 6. [Bug reports](#bug-reports)
41 
42 ## Building the library and applications
43 
44 ### Prerequisites
45 
46  1. [CMake](https://cmake.org) version 3.5 or higher.
47  2. [Git](https://git-scm.com/).
48  3. [Perl](https://www.perl.org/).
49  4. For x86 targets, [yasm](http://yasm.tortall.net/), which is preferred, or a
50     recent version of [nasm](http://www.nasm.us/).
51  5. Building the documentation requires [doxygen](http://doxygen.org).
52  6. Building the unit tests requires [Python](https://www.python.org/).
53  7. Emscripten builds require the portable
54    [EMSDK](https://kripken.github.io/emscripten-site/index.html).
55 
56 ### Get the code
57 
58 The AV1 library source code is stored in the Alliance for Open Media Git
59 repository:
60 
61 ~~~
62     $ git clone https://aomedia.googlesource.com/aom
63     # By default, the above command stores the source in the aom directory:
64     $ cd aom
65 ~~~
66 
67 ### Basic build
68 
69 CMake replaces the configure step typical of many projects. Running CMake will
70 produce configuration and build files for the currently selected CMake
71 generator. For most systems the default generator is Unix Makefiles. The basic
72 form of a makefile build is the following:
73 
74 ~~~
75     $ cmake path/to/aom
76     $ make
77 ~~~
78 
79 The above will generate a makefile build that produces the AV1 library and
80 applications for the current host system after the make step completes
81 successfully. The compiler chosen varies by host platform, but a general rule
82 applies: On systems where cc and c++ are present in $PATH at the time CMake is
83 run the generated build will use cc and c++ by default.
84 
85 ### Configuration options
86 
87 The AV1 codec library has a great many configuration options. These come in two
88 varieties:
89 
90  1. Build system configuration options. These have the form `ENABLE\_FEATURE`.
91  2. AV1 codec configuration options. These have the form `CONFIG\_FEATURE`.
92 
93 Both types of options are set at the time CMake is run. The following example
94 enables ccache and disables the AV1 encoder:
95 
96 ~~~
97     $ cmake path/to/aom -DENABLE\_CCACHE=1 -DCONFIG\_AV1\_ENCODER=0
98     $ make
99 ~~~
100 
101 The available configuration options are too numerous to list here. Build system
102 configuration options can be found at the top of the CMakeLists.txt file found
103 in the root of the AV1 repository, and AV1 codec configuration options can
104 currently be found in the file `build/cmake/aom\_config\_defaults.cmake`.
105 
106 ### Dylib builds
107 
108 A dylib (shared object) build of the AV1 codec library can be enabled via the
109 CMake built in variable `BUILD\_SHARED\_LIBS`:
110 
111 ~~~
112     $ cmake path/to/aom -DBUILD\_SHARED\_LIBS=1
113     $ make
114 ~~~
115 
116 This is currently only supported on non-Windows targets.
117 
118 ### Debugging
119 
120 Depending on the generator used there are multiple ways of going about
121 debugging AV1 components. For single configuration generators like the Unix
122 Makefiles generator, setting `CMAKE\_BUILD\_TYPE` to Debug is sufficient:
123 
124 ~~~
125     $ cmake path/to/aom -DCMAKE\_BUILD\_TYPE=Debug
126 ~~~
127 
128 For Xcode, mainly because configuration controls for Xcode builds are buried two
129 configuration windows deep and must be set for each subproject within the Xcode
130 IDE individually, `CMAKE\_CONFIGURATION\_TYPES` should be set to Debug:
131 
132 ~~~
133     $ cmake path/to/aom -G Xcode -DCMAKE\_CONFIGURATION\_TYPES=Debug
134 ~~~
135 
136 For Visual Studio the in-IDE configuration controls should be used. Simply set
137 the IDE project configuration to Debug to allow for stepping through the code.
138 
139 In addition to the above it can sometimes be useful to debug only C and C++
140 code. To disable all assembly code and intrinsics set `AOM\_TARGET\_CPU` to
141 generic at generation time:
142 
143 ~~~
144     $ cmake path/to/aom -DAOM\_TARGET\_CPU=generic
145 ~~~
146 
147 ### Cross compiling
148 
149 For the purposes of building the AV1 codec and applications and relative to the
150 scope of this guide, all builds for architectures differing from the native host
151 architecture will be considered cross compiles. The AV1 CMake build handles
152 cross compiling via the use of toolchain files included in the AV1 repository.
153 The toolchain files available at the time of this writing are:
154 
155  - arm64-ios.cmake
156  - arm64-linux-gcc.cmake
157  - arm64-mingw-gcc.cmake
158  - armv7-ios.cmake
159  - armv7-linux-gcc.cmake
160  - armv7-mingw-gcc.cmake
161  - armv7s-ios.cmake
162  - mips32-linux-gcc.cmake
163  - mips64-linux-gcc.cmake
164  - x86-ios-simulator.cmake
165  - x86-linux.cmake
166  - x86-macos.cmake
167  - x86-mingw-gcc.cmake
168  - x86\(\backslash\)\_64-ios-simulator.cmake
169  - x86\(\backslash\)\_64-mingw-gcc.cmake
170 
171 The following example demonstrates use of the x86-macos.cmake toolchain file on
172 a x86\(\backslash\)\_64 MacOS host:
173 
174 ~~~
175     $ cmake path/to/aom \(\backslash\)
176       -DCMAKE\_TOOLCHAIN\_FILE=path/to/aom/build/cmake/toolchains/x86-macos.cmake
177     $ make
178 ~~~
179 
180 To build for an unlisted target creation of a new toolchain file is the best
181 solution. The existing toolchain files can be used a starting point for a new
182 toolchain file since each one exposes the basic requirements for toolchain files
183 as used in the AV1 codec build.
184 
185 As a temporary work around an unoptimized AV1 configuration that builds only C
186 and C++ sources can be produced using the following commands:
187 
188 ~~~
189     $ cmake path/to/aom -DAOM\_TARGET\_CPU=generic
190     $ make
191 ~~~
192 
193 In addition to the above it's important to note that the toolchain files
194 suffixed with gcc behave differently than the others. These toolchain files
195 attempt to obey the $CROSS environment variable.
196 
197 ### Sanitizers
198 
199 Sanitizer integration is built-in to the CMake build system. To enable a
200 sanitizer, add `-DSANITIZE=<type>` to the CMake command line. For example, to
201 enable address sanitizer:
202 
203 ~~~
204     $ cmake path/to/aom -DSANITIZE=address
205     $ make
206 ~~~
207 
208 Sanitizers available vary by platform, target, and compiler. Consult your
209 compiler documentation to determine which, if any, are available.
210 
211 ### Microsoft Visual Studio builds
212 
213 Building the AV1 codec library in Microsoft Visual Studio is supported. Visual
214 Studio 2015 (14.0) or later is required. The following example demonstrates
215 generating projects and a solution for the Microsoft IDE:
216 
217 ~~~
218     # This does not require a bash shell; command.exe is fine.
219     $ cmake path/to/aom -G "Visual Studio 15 2017"
220 ~~~
221 
222 NOTE: The build system targets Windows 7 or later by compiling files with
223 `-D\_WIN32\_WINNT=0x0601`.
224 
225 ### Xcode builds
226 
227 Building the AV1 codec library in Xcode is supported. The following example
228 demonstrates generating an Xcode project:
229 
230 ~~~
231     $ cmake path/to/aom -G Xcode
232 ~~~
233 
234 ### Emscripten builds
235 
236 Building the AV1 codec library with Emscripten is supported. Typically this is
237 used to hook into the AOMAnalyzer GUI application. These instructions focus on
238 using the inspector with AOMAnalyzer, but all tools can be built with
239 Emscripten.
240 
241 It is assumed here that you have already downloaded and installed the EMSDK,
242 installed and activated at least one toolchain, and setup your environment
243 appropriately using the emsdk\(\backslash\)\_env script.
244 
245 1. Download [AOMAnalyzer](https://people.xiph.org/~mbebenita/analyzer/).
246 
247 2. Configure the build:
248 
249 ~~~
250     $ cmake path/to/aom \(\backslash\)
251         -DENABLE\_CCACHE=1 \(\backslash\)
252         -DAOM\_TARGET\_CPU=generic \(\backslash\)
253         -DENABLE\_DOCS=0 \(\backslash\)
254         -DENABLE\_TESTS=0 \(\backslash\)
255         -DCONFIG\_ACCOUNTING=1 \(\backslash\)
256         -DCONFIG\_INSPECTION=1 \(\backslash\)
257         -DCONFIG\_MULTITHREAD=0 \(\backslash\)
258         -DCONFIG\_RUNTIME\_CPU\_DETECT=0 \(\backslash\)
259         -DCONFIG\_WEBM\_IO=0 \(\backslash\)
260         -DCMAKE\_TOOLCHAIN\_FILE=path/to/emsdk-portable/.../Emscripten.cmake
261 ~~~
262 
263 3. Build it: run make if that's your generator of choice:
264 
265 ~~~
266     $ make inspect
267 ~~~
268 
269 4. Run the analyzer:
270 
271 ~~~
272     # inspect.js is in the examples sub directory of the directory in which you
273     # executed cmake.
274     $ path/to/AOMAnalyzer path/to/examples/inspect.js path/to/av1/input/file
275 ~~~
276 
277 ### Extra build flags
278 
279 Three variables allow for passing of additional flags to the build system.
280 
281 - AOM\(\backslash\)\_EXTRA\(\backslash\)\_C\(\backslash\)\_FLAGS
282 - AOM\(\backslash\)\_EXTRA\(\backslash\)\_CXX\(\backslash\)\_FLAGS
283 - AOM\(\backslash\)\_EXTRA\(\backslash\)\_EXE\(\backslash\)\_LINKER\(\backslash\)\_FLAGS
284 
285 The build system attempts to ensure the flags passed through the above variables
286 are passed to tools last in order to allow for override of default behavior.
287 These flags can be used, for example, to enable asserts in a release build:
288 
289 ~~~
290     $ cmake path/to/aom \(\backslash\)
291         -DCMAKE\_BUILD\_TYPE=Release \(\backslash\)
292         -DAOM\_EXTRA\_C\_FLAGS=-UNDEBUG \(\backslash\)
293         -DAOM\_EXTRA\_CXX\_FLAGS=-UNDEBUG
294 ~~~
295 
296 ## Testing the AV1 codec
297 
298 ### Testing basics
299 
300 There are several methods of testing the AV1 codec. All of these methods require
301 the presence of the AV1 source code and a working build of the AV1 library and
302 applications.
303 
304 #### 1. Unit tests:
305 
306 The unit tests can be run at build time:
307 
308 ~~~
309     # Before running the make command the LIBAOM\_TEST\_DATA\_PATH environment
310     # variable should be set to avoid downloading the test files to the
311     # cmake build configuration directory.
312     $ cmake path/to/aom
313     # Note: The AV1 CMake build creates many test targets. Running make
314     # with multiple jobs will speed up the test run significantly.
315     $ make runtests
316 ~~~
317 
318 #### 2. Example tests:
319 
320 The example tests require a bash shell and can be run in the following manner:
321 
322 ~~~
323     # See the note above about LIBAOM\_TEST\_DATA\_PATH above.
324     $ cmake path/to/aom
325     $ make
326     # It's best to build the testdata target using many make jobs.
327     # Running it like this will verify and download (if necessary)
328     # one at a time, which takes a while.
329     $ make testdata
330     $ path/to/aom/test/examples.sh --bin-path examples
331 ~~~
332 
333 #### 3. Encoder tests:
334 
335 When making a change to the encoder run encoder tests to confirm that your
336 change has a positive or negligible impact on encode quality. When running these
337 tests the build configuration should be changed to enable internal encoder
338 statistics:
339 
340 ~~~
341     $ cmake path/to/aom -DCONFIG\_INTERNAL\_STATS=1
342     $ make
343 ~~~
344 
345 The repository contains scripts intended to make running these tests as simple
346 as possible. The following example demonstrates creating a set of baseline clips
347 for comparison to results produced after making your change to libaom:
348 
349 ~~~
350     # This will encode all Y4M files in the current directory using the
351     # settings specified to create the encoder baseline statistical data:
352     $ cd path/to/test/inputs
353     # This command line assumes that run\_encodes.sh, its helper script
354     # best\_encode.sh, and the aomenc you intend to test are all within a
355     # directory in your PATH.
356     $ run\_encodes.sh 200 500 50 baseline
357 ~~~
358 
359 After making your change and creating the baseline clips, you'll need to run
360 encodes that include your change(s) to confirm that things are working as
361 intended:
362 
363 ~~~
364     # This will encode all Y4M files in the current directory using the
365     # settings specified to create the statistical data for your change:
366     $ cd path/to/test/inputs
367     # This command line assumes that run\_encodes.sh, its helper script
368     # best\_encode.sh, and the aomenc you intend to test are all within a
369     # directory in your PATH.
370     $ run\_encodes.sh 200 500 50 mytweak
371 ~~~
372 
373 After creating both data sets you can use `test/visual\_metrics.py` to generate a
374 report that can be viewed in a web browser:
375 
376 ~~~
377     $ visual\_metrics.py metrics\_template.html "*stt" baseline mytweak \(\backslash\)
378       > mytweak.html
379 ~~~
380 
381 You can view the report by opening mytweak.html in a web browser.
382 
383 
384 ### IDE hosted tests
385 
386 By default the generated projects files created by CMake will not include the
387 runtests and testdata rules when generating for IDEs like Microsoft Visual
388 Studio and Xcode. This is done to avoid intolerably long build cycles in the
389 IDEs-- IDE behavior is to build all targets when selecting the build project
390 options in MSVS and Xcode. To enable the test rules in IDEs the
391 `ENABLE\_IDE\_TEST\_HOSTING` variable must be enabled at CMake generation time:
392 
393 ~~~
394     # This example uses Xcode. To get a list of the generators
395     # available, run cmake with the -G argument missing its
396     # value.
397     $ cmake path/to/aom -DENABLE\_IDE\_TEST\_HOSTING=1 -G Xcode
398 ~~~
399 
400 ### Downloading the test data
401 
402 The fastest and easiest way to obtain the test data is to use CMake to generate
403 a build using the Unix Makefiles generator, and then to build only the testdata
404 rule:
405 
406 ~~~
407     $ cmake path/to/aom -G "Unix Makefiles"
408     # 28 is used because there are 28 test files as of this writing.
409     $ make -j28 testdata
410 ~~~
411 
412 The above make command will only download and verify the test data.
413 
414 ### Adding a new test data file
415 
416 First, add the new test data file to the `aom-test-data` bucket of the
417 `aomedia-testing` project on Google Cloud Platform. You may need to ask someone
418 with the necessary access permissions to do this for you.
419 
420 NOTE: When a new test data file is added to the `aom-test-data` bucket, its
421 "Public access" is initially "Not public". We need to change its
422 "Public access" to "Public" by using the following
423 [`gsutil`](https://cloud.google.com/storage/docs/gsutil\_install) command:
424 ~~~
425     $ gsutil acl ch -g all:R gs://aom-test-data/test-data-file-name
426 ~~~
427 This command grants the `AllUsers` group READ access to the file named
428 "test-data-file-name" in the `aom-test-data` bucket.
429 
430 Once the new test data file has been added to `aom-test-data`, create a CL to
431 add the name of the new test data file to `test/test\_data\_util.cmake` and add
432 the SHA1 checksum of the new test data file to `test/test-data.sha1`. (The SHA1
433 checksum of a file can be calculated by running the `sha1sum` command on the
434 file.)
435 
436 ### Additional test data
437 
438 The test data mentioned above is strictly intended for unit testing.
439 
440 Additional input data for testing the encoder can be obtained from:
441 https://media.xiph.org/video/derf/
442 
443 ### Sharded testing
444 
445 The AV1 codec library unit tests are built upon gtest which supports sharding of
446 test jobs. Sharded test runs can be achieved in a couple of ways.
447 
448 #### 1. Running test\(\backslash\)\_libaom directly:
449 
450 ~~~
451    # Set the environment variable GTEST\_TOTAL\_SHARDS to control the number of
452    # shards.
453    $ export GTEST\_TOTAL\_SHARDS=10
454    # (GTEST shard indexing is 0 based).
455    $ seq 0 $(( $GTEST\_TOTAL\_SHARDS - 1 )) \(\backslash\)
456        | xargs -n 1 -P 0 -I\{\} env GTEST\_SHARD\_INDEX=\{\} ./test\_libaom
457 ~~~
458 
459 To create a test shard for each CPU core available on the current system set
460 `GTEST\_TOTAL\_SHARDS` to the number of CPU cores on your system minus one.
461 
462 #### 2. Running the tests via the CMake build:
463 
464 ~~~
465     # For IDE based builds, ENABLE\_IDE\_TEST\_HOSTING must be enabled. See
466     # the IDE hosted tests section above for more information. If the IDE
467     # supports building targets concurrently tests will be sharded by default.
468 
469     # For make and ninja builds the -j parameter controls the number of shards
470     # at test run time. This example will run the tests using 10 shards via
471     # make.
472     $ make -j10 runtests
473 ~~~
474 
475 The maximum number of test targets that can run concurrently is determined by
476 the number of CPUs on the system where the build is configured as detected by
477 CMake. A system with 24 cores can run 24 test shards using a value of 24 with
478 the `-j` parameter. When CMake is unable to detect the number of cores 10 shards
479 is the default maximum value.
480 
481 ## Coding style
482 
483 We are using the Google C Coding Style defined by the
484 [Google C++ Style Guide](https://google.github.io/styleguide/cppguide.html).
485 
486 The coding style used by this project is enforced with clang-format using the
487 configuration contained in the
488 [.clang-format](https://chromium.googlesource.com/webm/aom/+/master/.clang-format)
489 file in the root of the repository.
490 
491 You can download clang-format using your system's package manager, or directly
492 from [llvm.org](http://llvm.org/releases/download.html). You can also view the
493 [documentation](https://clang.llvm.org/docs/ClangFormat.html) on llvm.org.
494 Output from clang-format varies by clang-format version, for best results your
495 version should match the one used on Jenkins. You can find the clang-format
496 version by reading the comment in the `.clang-format` file linked above.
497 
498 Before pushing changes for review you can format your code with:
499 
500 ~~~
501     # Apply clang-format to modified .c, .h and .cc files
502     $ clang-format -i --style=file \(\backslash\)
503       $(git diff --name-only --diff-filter=ACMR '*.[hc]' '*.cc')
504 ~~~
505 
506 Check the .clang-format file for the version used to generate it if there is any
507 difference between your local formatting and the review system.
508 
509 Some Git installations have clang-format integration. Here are some examples:
510 
511 ~~~
512     # Apply clang-format to all staged changes:
513     $ git clang-format
514 
515     # Clang format all staged and unstaged changes:
516     $ git clang-format -f
517 
518     # Clang format all staged and unstaged changes interactively:
519     $ git clang-format -f -p
520 ~~~
521 
522 ## Submitting patches
523 
524 We manage the submission of patches using the
525 [Gerrit](https://www.gerritcodereview.com/) code review tool. This tool
526 implements a workflow on top of the Git version control system to ensure that
527 all changes get peer reviewed and tested prior to their distribution.
528 
529 ### Login cookie
530 
531 Browse to [AOMedia Git index](https://aomedia.googlesource.com/) and login with
532 your account (Gmail credentials, for example). Next, follow the
533 `Generate Password` Password link at the top of the page. You’ll be given
534 instructions for creating a cookie to use with our Git repos.
535 
536 ### Contributor agreement
537 
538 You will be required to execute a
539 [contributor agreement](http://aomedia.org/license) to ensure that the AOMedia
540 Project has the right to distribute your changes.
541 
542 ### Testing your code
543 
544 The testing basics are covered in the [testing section](#testing-the-av1-codec)
545 above.
546 
547 In addition to the local tests, many more (e.g. asan, tsan, valgrind) will run
548 through Jenkins instances upon upload to gerrit.
549 
550 ### Commit message hook
551 
552 Gerrit requires that each submission include a unique Change-Id. You can assign
553 one manually using git commit --amend, but it’s easier to automate it with the
554 commit-msg hook provided by Gerrit.
555 
556 Copy commit-msg to the `.git/hooks` directory of your local repo. Here's an
557 example:
558 
559 ~~~
560     $ curl -Lo aom/.git/hooks/commit-msg https://chromium-review.googlesource.com/tools/hooks/commit-msg
561 
562     # Next, ensure that the downloaded commit-msg script is executable:
563     $ chmod u+x aom/.git/hooks/commit-msg
564 ~~~
565 
566 See the Gerrit
567 [documentation](https://gerrit-review.googlesource.com/Documentation/user-changeid.html)
568 for more information.
569 
570 ### Upload your change
571 
572 The command line to upload your patch looks like this:
573 
574 ~~~
575     $ git push https://aomedia-review.googlesource.com/aom HEAD:refs/for/master
576 ~~~
577 
578 ### Incorporating reviewer comments
579 
580 If you previously uploaded a change to Gerrit and the Approver has asked for
581 changes, follow these steps:
582 
583 1. Edit the files to make the changes the reviewer has requested.
584 2. Recommit your edits using the --amend flag, for example:
585 
586 ~~~
587    $ git commit -a --amend
588 ~~~
589 
590 3. Use the same git push command as above to upload to Gerrit again for another
591    review cycle.
592 
593 In general, you should not rebase your changes when doing updates in response to
594 review. Doing so can make it harder to follow the evolution of your change in
595 the diff view.
596 
597 ### Submitting your change
598 
599 Once your change has been Approved and Verified, you can “submit” it through the
600 Gerrit UI. This will usually automatically rebase your change onto the branch
601 specified.
602 
603 Sometimes this can’t be done automatically. If you run into this problem, you
604 must rebase your changes manually:
605 
606 ~~~
607     $ git fetch
608     $ git rebase origin/branchname
609 ~~~
610 
611 If there are any conflicts, resolve them as you normally would with Git. When
612 you’re done, reupload your change.
613 
614 ### Viewing the status of uploaded changes
615 
616 To check the status of a change that you uploaded, open
617 [Gerrit](https://aomedia-review.googlesource.com/), sign in, and click My >
618 Changes.
619 
620 ## Support
621 
622 This library is an open source project supported by its community. Please
623 please email aomediacodec@jointdevelopment.kavi.com for help.
624 
625 ## Bug reports
626 
627 Bug reports can be filed in the Alliance for Open Media
628 [issue tracker](https://bugs.chromium.org/p/aomedia/issues/list).
\end{DoxyCodeInclude}
 